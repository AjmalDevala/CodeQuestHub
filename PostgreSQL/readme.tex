What Is PostgreSQL?  our local db SQL Shell password {mobile password:eg 5....0}

PostgreSQL (often referred to as Postgres) is a powerful, open-source,object-relational database management system (ORDBMS). 
It is known for its reliability, feature set, and standards compliance. 
PostgreSQL has been in development for over 30 years, making it a mature and robust database choice for a wide variety of applications,
from small-scale projects to large enterprise systems

% Conventions

The following conventions are used in the synopsis of a command:( brackets )([ and ]) indicate optional parts.
(Braces) ({ and }) and (vertical lines) (|) indicate that you must choose one alternative. 
Dots (...) mean that the preceding element can be repeated. All other symbols, including parentheses, should be taken literally.


% // ................BASIS METHOD..............................................................//

% Select query with LEFT/RIGHT/FULL JOINs on multiple tables

SELECT column, another_column, …
FROM mytable
INNER/LEFT/RIGHT/FULL JOIN another_table 
    ON mytable.id = another_table.matching_id
WHERE condition(s)
ORDER BY column, … ASC/DESC
LIMIT num_limit OFFSET num_offset;


% Users is the Table      (NAME ,Age , Email , Password,)

NAME | AGE| EMAIL      |PASSWORD
............................... 
Aaa  | 12 |  AAA@GMAIL | 12345
BBB  | 13 |  bbb@GMAIL | 12352
CCC  | 15 |  ccc@GMAIL | 12523

% ........................ Basic SQL Queries:.....................................%

{Select all} columns from the Users table:

% SELECT * FROM Users;

This retrieves all records and columns from the Users table.

Select {specific columns }(e.g., name) from the Users table:

% SELECT name FROM Users;

This retrieves only the name column from the Users table.

{Filter records }using the WHERE clause:

% SELECT * FROM Users WHERE Age > 25;

Retrieves users who are older than 25.

Use DISTINCT to remove {duplicates}:

% SELECT DISTINCT Email FROM Users;

Retrieves unique email addresses from the Users table.

Order the results using {ORDER BY:} // ascending (ASC) designing(DESC)


% SELECT * FROM Users ORDER BY Age ASC;
Retrieves all users and orders them by Age in ascending order.

% ...............................Intermediate SQL Queries............................%

{Insert} a new record into the Users table: //INSERT INTO table name ('datas') values ('datas')

% INSERT INTO Users (name, age, email, password)
% VALUES ('John Doe', 30, 'john@example.com', 'hashed_password');

{Update} an existing record:

% UPDATE Users
% SET email = 'john.doe@example.com'
% WHERE name = 'John Doe';


{Delete} a record from the Users table:

% DELETE FROM Users
% WHERE name = 'John Doe';

Use LIKE for pattern matching (wildcards):


% SELECT * FROM Users
% WHERE email LIKE '%example.com';

Retrieves all users whose email ends with example.com.

% ......................................Advanced SQL Queries:.................................%



% Use aggregate functions (COUNT, SUM, AVG, etc.):

Count all users:

% SELECT COUNT(*) FROM Users;

Get the average age of users:

% SELECT AVG(Age) FROM Users;

Group records using GROUP BY and aggregate functions:


% SELECT Age, COUNT(*) AS num_users
% FROM Users
% GROUP BY Age;

This groups users by their age and counts the number of users in each age group.

Join two tables (e.g.,{ Posts} and Users): // post table also There


% SELECT Users.name, Posts.title
% FROM Users
% JOIN Posts ON Users.id = Posts.user_id;


This retrieves the names of users and their corresponding blog post titles.

Subqueries (nested queries):


% SELECT * FROM Users
% WHERE Age > (SELECT AVG(Age) FROM Users);

Retrieves users whose age is greater than the average age of all users.

Use {CASE for conditional} logic in SQL:


% SELECT name, Age,
% CASE
%     WHEN Age < 18 THEN 'Minor'
%     WHEN Age BETWEEN 18 AND 64 THEN 'Adult'
%     ELSE 'Senior'
% END AS Age_Group
% FROM Users;

Use transactions for multiple queries:


% BEGIN;
% UPDATE Users SET Age = Age + 1 WHERE id = 1;
% DELETE FROM Users WHERE id = 2;
% COMMIT;

These queries can give your readers a good mix of basic to advanced SQL operations, 
helping them understand core SQL functionalities as well as more complex operations such as joins,
 subqueries, and transactions.
