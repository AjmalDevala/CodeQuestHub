<!-- .......................What is Node.js? -->

Node.js is a JavaScript runtime built on Chrome's V8 JavaScript engine.
It allows developers to use JavaScript to build server-side applications.
Non-blocking and asynchronous nature, making it efficient for I/O-heavy operations.


<!-- .........................................Key Features: -->

<!-- Asynchronous & Event-Driven: -->

All APIs of Node.js are asynchronous, meaning that the server doesn't wait for an API to return data.
It uses an event-driven architecture.

<!-- Single-Threaded but Scalable: -->

Node.js operates on a single thread using the event loop, but can handle thousands of concurrent connections thanks to asynchronous I/O.


    Fast Execution:

Built on Chrome’s V8 JavaScript engine, Node.js compiles JavaScript into machine code for faster execution.

    NPM (Node Package Manager):

Comes with Node.js, providing a vast collection of libraries and modules for different tasks.

    Common Use Cases:

Web Servers: Node.js is excellent for real-time applications like chat applications or live updates.

    APIs: Build RESTful APIs and microservices.

Streaming Applications: Ideal for data streaming applications like video platforms.


<!--........................................ Basic Example (Creating a Server): -->

<!-- javascript -->

const http = require('http');

<!-- // Create a server -->

 http.createServer((req, res) => {
    res.writeHead(200, {'Content-Type': 'text/plain'});
    res.end('Hello World\n');
}).listen(3000);   

console.log('Server running at http://localhost:3000/');



<!--............................. Explanation: -->

This code creates a basic HTTP server that listens on port 3000 and responds with "Hello World".


<!-- Important Modules in Node.js: -->


{HTTP Module: }            For creating web servers and handling HTTP requests.
File System (fs) Module:   For interacting with the file system.
{Path Module: }            For working with file and directory paths.
{Express.js (popular framework):}  Used for building robust APIs and web applications.


% Event-Driven Model:


Node.js uses an event-driven model, where the server initiates a callback function whenever an event occurs (e.g., data received, server request).
Example:

javascript


const events = require('events');
const eventEmitter = new events.EventEmitter();

// Create an event handler


const myEventHandler = () => {
    console.log('I hear a scream!');
}

// Assign the event handler to an event

eventEmitter.on('scream', myEventHandler);

// Fire the 'scream' event


eventEmitter.emit('scream');

% Advantages of Node.js:

Fast: Non-blocking I/O and V8 engine ensure fast execution.
Single Codebase: Use JavaScript on both client and server.
Large Ecosystem: Huge number of libraries through NPM.


% Disadvantages of Node.js:


Single-threaded: May struggle with CPU-intensive tasks.
Callback Hell: Managing many nested callbacks can lead to complicated code (mitigated by promises and async/await).