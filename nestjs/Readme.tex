% .........................What is  nestjs .....................................

Nest (NestJS) is a framework for building efficient, scalable Node.js server-side applications. 
It uses progressive JavaScript, is built with and fully supports TypeScript (yet still enables developers to code in pure JavaScript) 
and combines elements of OOP (Object Oriented Programming), FP (Functional Programming), and FRP (Functional Reactive Programming).

%  ......................   Basic stucture ..............
app.controller.ts	- A basic controller with a single route.
app.controller.spec.ts	- The unit tests for the controller.
app.module.ts -	The root module of the application.
app.service.ts	- A basic service with a single method.
main.ts -	The entry file of the application which uses the core function NestFactory to create a Nest application instance.


% ....................  What is DTO ...........................

DTO (Data Transfer Object) in NestJS defines the shape of the data and ensures data validation and security.
It is widely used in the service and controller layers of an API to structure and validate incoming data.


% ............................what is nestjs...........................................//

NestJS is a progressive Node.js framework for building efficient, scalable, and maintainable server-side applications.
It is built with and fully supports TypeScript, although it also allows developers to use JavaScript. 
NestJS draws inspiration from Angular, with a modular architecture that provides structure for both small and large-scale applications.

%............................ Key Features of NestJS: .................................//

% Modular Architecture:

NestJS applications are organized into modules. Each feature of the application is typically encapsulated in its own module,
making the app more maintainable and scalable.

% Dependency Injection:

NestJS uses a powerful dependency injection system, which makes it easy to manage services and their dependencies.

% TypeScript Support:

Built from the ground up to work with TypeScript, which provides static typing and improves development speed and error prevention.

% Supports MVC Pattern:

Like traditional MVC frameworks, NestJS organizes code into Controllers, Services, and Providers for clean separation of concerns.

% Compatibility with Other Libraries:

You can easily integrate third-party libraries like Express.js, Fastify, TypeORM, Mongoose, and more, 
providing flexibility in designing applications.

% Middleware and Guards:
NestJS has support for middleware, guards, pipes, and interceptors, 
giving you fine-grained control over request/response handling, security, and validation.

% Built-in Support for REST and GraphQL APIs:

It’s easy to create RESTful APIs with NestJS. 
It also has first-class support for building GraphQL APIs.

% Cross-platform Development:

NestJS can be used to build not only web applications but also microservices, 
real-time applications (with WebSockets), and more.

% Rich Ecosystem:

NestJS has a rich ecosystem of built-in modules for common tasks such as authentication, \
database integration, caching, logging, and more.

% Testing Support:

NestJS promotes testing and provides easy setup for unit, integration, 
and end-to-end testing using popular tools like Jest.

% ........................................................Use Cases of NestJS:...............................................//

RESTful APIs:       Build highly scalable and maintainable REST APIs with well-defined architecture.
Microservices:      Suitable for building microservices-based architectures.
GraphQL APIs:       NestJS has built-in support for GraphQL, making it a great choice for GraphQL-based projects.
Real-time Applications:    With WebSocket support, you can build real-time apps like chat applications or live notifications.
Enterprise Applications:   The modular design and scalability make it a solid choice for large-scale enterprise applications.
Comparison to Other Frameworks:    NestJS vs Express.js: While NestJS uses Express.js under the hood by default, it adds a lot of structure and TypeScript support, 
                                   which can make it easier to manage large applications. Express is more minimal, while NestJS provides more tools out of the box.
NestJS vs Fastify:   Fastify can also be used as a backend framework in NestJS, which can result in faster performance for specific use cases.
NestJS vs Angular:   If you're familiar with Angular, NestJS will feel familiar in terms of its modular architecture and dependency injection system, 
                     even though it's for backend development.

% //..................... Over all summary ..............................//

In summary, NestJS is a powerful framework that combines the efficiency of Node.js with the structure of a full-fledged modern framework, 
making it ideal for building reliable, scalable server-side applications.